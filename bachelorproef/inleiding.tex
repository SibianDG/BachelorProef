%%=============================================================================
%% Inleiding
%%=============================================================================

\chapter{\IfLanguageName{dutch}{Inleiding}{Introduction}}
\label{ch:inleiding}

De veroudering van de bevolking in de Vlaamse steden en gemeenten zet zich in de komende decennia verder ~\autocite{StatistiekVlaanderen2018}.
Volgens de voorspellingen zou tegen 2033 25\% van de bevolking een 65-plusser zijn.

Tegelijk is het woord `waardigheid' actueler dan ooit.
Na de schrijnende omstandigheden van de Tweede Wereldoorlog stond dat woord centraal bij het opstellen van het Verdrag van de Verenigde Naties (1945), de Universele Verklaring van de Rechten van de mens (1948) en de grondrechten van de Europese Unie (2000).
Die basiswaarde vinden we ook terug bij het Europese en Belgische zorgbeleid. Ouderen mogen niet gediscrimineerd worden op vlak van leeftijd. Tevens mogen ze ook niet op een kinderlijke, betuttelende of onvriendelijke wijze aangesproken worden en moeten ze met respect bejegend worden~\autocite{Campens2021}.

Hoe meer ouderen er in de samenleving zijn, hoe groter het zorgaanbod dat de samenleving moet organiseren voor deze leeftijdscategorie.
Die zorgverleners, maar evengoed familie, weten niet altijd goed hoe ze moeten omgaan met ouderen.
Wanneer een jonger persoon op een andere manier spreekt tegen een senior dan tegen een leeftijdsgenoot, spreken we over \textit{elderspeak}. \textcite{Williams2011} omschrijft \textit{elderspeak} als volgt: ``Elderspeak is a common intergenerational speech style used by younger persons in communication with older adults in a variety of community and health care settings. Based on negative stereotypes of older adults as less competent communicators, younger speakers (in this case nursing home staff) modify their communication with nursing home residents by simplifying the vocabulary and grammar and by adding clarifications such as repetitions and altered prosody.'' Kortom, jongere mensen passen hun communicatie aan, door hun woordenschat te vereenvoudigen en verduidelijkingen toe te voegen zoals herhalingen, evenals hun prosodie (de fonologie, het ritme, de klemtoon en de intonatie van de stem). Om \textit{elderspeak} te bestrijden, gaven \textcite{Wick2007} een paar tips mee in hun onderzoek.
Enkele van die tips gingen als volgt: spreek mensen aan zoals ze wensen aangesproken te worden, vraag om ze aan te spreken met de voornaam, vermijd troetelnamen, wees bewust van non-verbaal gedrag, verhoog uw stemvolume enkel wanneer uw gesprekspartner hardhorig is, herhaal alleen uw zin als uw gesprekspartner het niet begrepen heeft, vermijd korte, langzame en makkelijke zinnen, vermijd verkleinwoorden en hanteer beleefd taalgebruik.

Naast \textit{elderspeak} vormt het fenomeen \textit{nursery tone} een extra uitdaging. Dit verwijst naar de situatie waarbij iemand de toonhoogte aan het einde van de zin standaard verhoogt zoals bij communicatie met jonge kinderen.

\textcite{Beeckman2021} en \textcite{Standaert2021} werkten vorig jaar in hun bachelorproef al aan de opzet van dit onderwerp en dit onderzoek zal verder werken op hun gelegde basis.
Sommige stukken programmacode van hen zullen gebruikt worden om zo een beter model op te stellen.
Zij haalden zelf ook verbeterpunten aan en moeilijkheden die, hopelijk, op te lossen zijn. Wat het verschil zal zijn tussen hun eindwerken en dit eindwerk wordt toegelicht in~\ref{sec:state-of-the-art}.

De nog steeds relevante onderzoeksvraag van dit onderwerp is: ``Kan \textit{elderspeak} gedetecteerd worden door Artificiële Intelligentie en is het nuttig om AI toe te passen in de praktijk?''. Een bijkomende onderzoeksvraag is: ``Kan \textit{nursery tone} gedetecteerd worden door Artificiële Intelligentie?''. De laatste onderzoeksvraag moet een antwoord bieden op: ``Kan de applicatie nuttig zijn voor zorgkundigen?''

\section{\IfLanguageName{dutch}{Probleemstelling}{Problem Statement}}
\label{sec:probleemstelling}

Dit onderzoek heeft als doel een meerwaarde te betekenen voor de oudere mensen in woonzorgcentra en ziekenhuizen, maar ook nog de ouderen die zelfstandig thuis wonen. Ze vinden het namelijk helemaal niet aangenaam om aangesproken te worden als kleine kinderen. Deze applicatie zal \textit{elderspeak} herkennen en aangeven welke elementen van \textit{elderspeak} aanwezig zijn. Het zijn vooral de verpleegkundigen, artsen en ander zorgpersoneel, maar ook familieleden die zich bewust moeten zijn van de manier waarop ze tegen ouderen praten.

\section{\IfLanguageName{dutch}{Onderzoeksvraag}{Research question}}
\label{sec:onderzoeksvraag}

De onderzoeksvraag die bij dit eindwerk hoort, is: ``Kan \textit{elderspeak} gedetecteerd worden door Artificiële Intelligentie en kan dit toegepast worden in de praktijk?''. Een bijkomende onderzoeksvraag is: ``Kan \textit{nursery tone} gedetecteerd worden door Artificiële Intelligentie?''. Maar enkel dit beantwoorden is uiteraard niet genoeg.
Aan de hand van de volgende deelvragen kan er een duidelijker en uitgebreider antwoord geformuleerd worden:

\begin{itemize}
    \item Welk type Artificiële Intelligentie past het beste bij deze opstelling?
    \item Welk type model van \textit{machine learning} of \textit{deep learning} werkt het beste per eigenschap?
    \item Kan je achtergrondlawaai wegfilteren en hoe precies?
    \item Zal spraakherkenning lukken met de gratis beschikbare softwarebibliotheken?
    \item Hoe zet je een ``Flask'' server op waar je \textit{webrequests} naar stuurt? En hoe verbind je daar een model mee?
    \item Is het nuttig voor zorgkundigen om deze applicatie in gebruik te nemen?
\end{itemize}

\section{\IfLanguageName{dutch}{Onderzoeksdoelstelling}{Research objective}}
\label{sec:onderzoeksdoelstelling}

Deze bachelorproef heeft als doel om een basiswebsite aan te bieden die dient als \textit{PoC} of \textit{proof-of-concept}. Die basisapplicatie vraagt eerst om te praten zoals je zou doen tegen je vrienden. Nadien wordt er gevraagd om te praten zoals je zou doen bij slechthorende ouderen. Ter ondersteuning zal er een foto getoond worden van iemand uit een woonzorgcentrum. De applicatie analyseert dan de audiosamples en geeft aan welke kenmerken er aanwezig waren van \textit{elderspeak}.
Die kenmerken en de reden dat het model oordeelt dat die aanwezig zijn, zijn vergaard in het literatuuronderzoek.

\section{\IfLanguageName{dutch}{Opzet van deze bachelorproef}{Structure of this bachelor thesis}}
\label{sec:opzet-bachelorproef}

% Het is gebruikelijk aan het einde van de inleiding een overzicht te
% geven van de opbouw van de rest van de tekst. Deze sectie bevat al een aanzet
% die je kan aanvullen/aanpassen in functie van je eigen tekst.

De rest van deze bachelorproef is als volgt opgebouwd:

In Hoofdstuk~\ref{ch:stand-van-zaken} wordt een overzicht gegeven van de stand van zaken binnen het onderzoeksdomein, op basis van een literatuurstudie.

In Hoofdstuk~\ref{ch:methodologie} wordt de methodologie toegelicht en worden de gebruikte onderzoekstechnieken besproken om een antwoord te kunnen formuleren op de onderzoeksvragen.

In Hoofdstuk~\ref{ch:resultaten} worden de resultaten beschreven van deze bachelorproef zoals de verzamelde data, de resultaten na het testen, hoe goed een rollenspel werkt en of er aan de \textit{requirements} voldaan zijn.

In Hoofdstuk~\ref{ch:vervolg} wordt er beschreven hoe de programmacode gedeeld kan worden, hoe de applicatie gehost kan worden en hoe men in het vervolg beter kan testen.

In Hoofdstuk~\ref{ch:conclusie} ten slotte, wordt de conclusie gegeven en een antwoord geformuleerd op de onderzoeksvragen. Daarbij wordt ook een aanzet gegeven voor toekomstig onderzoek binnen dit domein.