%%=============================================================================
%% Conclusie
%%=============================================================================

\chapter{Conclusie}
\label{ch:conclusie}

% TODO: Trek een duidelijke conclusie, in de vorm van een antwoord op de
% onderzoeksvra(a)g(en). Wat was jouw bijdrage aan het onderzoeksdomein en
% hoe biedt dit meerwaarde aan het vakgebied/doelgroep?
% Reflecteer kritisch over het resultaat. In Engelse teksten wordt deze sectie
% ``Discussion'' genoemd. Had je deze uitkomst verwacht? Zijn er zaken die nog
% niet duidelijk zijn?
% Heeft het onderzoek geleid tot nieuwe vragen die uitnodigen tot verder
%onderzoek?

%TODO: je moet niet op alles AI toepassen

\color{blue}
TODO: Trek een duidelijke conclusie, in de vorm van een antwoord op de
 onderzoeksvra(a)g(en). Wat was jouw bijdrage aan het onderzoeksdomein en
 hoe biedt dit meerwaarde aan het vakgebied/doelgroep?
 Reflecteer kritisch over het resultaat. In Engelse teksten wordt deze sectie
 ``Discussion'' genoemd. Had je deze uitkomst verwacht? Zijn er zaken die nog
 niet duidelijk zijn?
 Heeft het onderzoek geleid tot nieuwe vragen die uitnodigen tot verder
onderzoek?


\begin{itemize}
    \item Welk type Artificiële Intelligentie past het beste bij deze opstelling?
    \item Welk type model van \textit{machine learning} of \textit{deep learning} werkt het beste per eigenschap?
    \item Kan je achtergrond lawaai wegfilteren en hoe precies?
    \item Zal spraakherkenning lukken met de gratis beschikbare softwarebibliotheken?
    \item Hoe zet je een ``Flask'' server op waar je \textit{webrequests} naar stuurt? En hoe verbind je daar een model mee?
    \item Is het nuttig om zo deze applicatie in gebruik te nemen voor zorgkundigen?
\end{itemize}

\color{black}

Met deze bachelorproef kan er besloten worden dat \textit{elderspeak} of \textit{nursery tone} onrechtstreeks kan gedetecteerd worden met artificiële intelligentie. De kunstmatige intelligentie die gebruikt wordt, is die dat verwerkt zit in de Google Speech Recognition-API.
Het type model dat Google gebruikt is natuurlijk niet zomaar online te vinden omdat anders andere mensen dit kunnen kopiëren, maar wat er wel geweten is, is dat Google aan \textit{natural language processing} doet.


\section{Discussie}

