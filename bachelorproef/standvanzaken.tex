\chapter{\IfLanguageName{dutch}{Stand van zaken}{State of the art}}
\label{ch:stand-van-zaken}

% Tip: Begin elk hoofdstuk met een paragraaf inleiding die beschrijft hoe
% dit hoofdstuk past binnen het geheel van de bachelorproef. Geef in het
% bijzonder aan wat de link is met het vorige en volgende hoofdstuk.

% Pas na deze inleidende paragraaf komt de eerste sectiehoofding.


\color{blue}
Dit hoofdstuk bevat je literatuurstudie. De inhoud gaat verder op de inleiding, maar zal het onderwerp van de bachelorproef *diepgaand* uitspitten. De bedoeling is dat de lezer na lezing van dit hoofdstuk helemaal op de hoogte is van de huidige stand van zaken (state-of-the-art) in het onderzoeksdomein. Iemand die niet vertrouwd is met het onderwerp, weet nu voldoende om de rest van het verhaal te kunnen volgen, zonder dat die er nog andere informatie moet over opzoeken \autocite{Pollefliet2011}.

Je verwijst bij elke bewering die je doet, vakterm die je introduceert, enz. naar je bronnen. In \LaTeX{} kan dat met het commando \texttt{$\backslash${textcite\{\}}} of \texttt{$\backslash${autocite\{\}}}. Als argument van het commando geef je de ``sleutel'' van een ``record'' in een bibliografische databank in het Bib\LaTeX{}-formaat (een tekstbestand). Als je expliciet naar de auteur verwijst in de zin, gebruik je \texttt{$\backslash${}textcite\{\}}.
Soms wil je de auteur niet expliciet vernoemen, dan gebruik je \texttt{$\backslash${}autocite\{\}}. In de volgende paragraaf een voorbeeld van elk.

\textcite{Knuth1998} schreef een van de standaardwerken over sorteer- en zoekalgoritmen. Experten zijn het erover eens dat cloud computing een interessante opportuniteit vormen, zowel voor gebruikers als voor dienstverleners op vlak van informatietechnologie~\autocite{Creeger2009}.

\color{black}

\section{Verkennend onderzoek}
\label{sec:verkennend-onderzoek}
``Kan \textit{elderspeak} gedetecteerd worden door Artificiële Intelligentie en kan dit toegepast worden in de praktijk?'', is de centrale onderzoeksvraag, maar daarvoor moeten we twee begrippen goed uitleggen en begrijpen om te kunnen staven of dit wel degelijk mogelijk is.
Er zal dus eerst beschreven worden wat \textit{elderspeak} precies is. Waarom vinden ouderen dat niet leuk? Wat zijn de eigenschappen en hoe kan je het voorkomen?
Daarnaast moeten we ook begrijpen wat Artificiële Intelligentie is. De reden hiervoor is dat er verstaan moet worden wat dat is, welke types er zijn en hoe dit gebruikt werd het eindresultaat.

\section{Elderspeak}

\subsection{Wat is \textit{Elderspeak}?}

Het begrip \textit{elderspeak}, ook \textit{secondary babytalk} genoemd, kent verschillende definities. Kemper, Finter‐Urczyk, Ferrell, Harden en Billington (1998) TODO source omschrijft het begrip als volgt:
"Elderspeak is a simplified speech register with exaggerated pitch and intonation, simplified grammar, limited vocabulary and slow rate of delivery.”

Daarnaast beschrijft Williams (2011) TODO het begrip als het volgende:
“Elderspeak is a common intergenerational speech style used by younger persons in communication with older adults in a variety of community and health care settings. Based on negative stereotypes of older adults as less competent communicators, younger speakers (in this case nursing home staff) modify their communication with nursing home residents by simplifying the vocabulary and grammar and by adding clarifications such as repetitions and altered prosody.”

\subsection{Wat zijn de kenmerken?}
Deze kenmerken komen overeen met een communicatiestijl die men hanteert wanneer men tegen (afhankelijke) kinderen praat. Vandaar dat \textit{elderspeak} ook wel als \textit{secondary baby} talk wordt benoemd.

\begin{itemize}
    \item Langzaam spreken
    \item Verhoogde toonhoogte
    \item Verhoogd stemvolume
    \item Overdreven intonatie
    \item Vereenvoudigd woordgebruik, gebruik van verkleinwoorden en/of ongepaste bijnamen of troetelnamen
    \item Verminderde grammaticale complexiteit (bv. voornamelijk enkelvoudige zinnen)
    \item Gebruik van collectieve voornaamwoorden (bijvoorbeeld “we” in plaats van “jij”)
    \item Veelvuldig gebruik van (bevestigende) tussenwerpsels (zoals “hé” of voilà”)
    \item Gewijzigd non-verbaal gedrag (bv. langdurig oogcontact, extra gebaren, te dichtbij komen)
    \item Veelvuldige verduidelijking en herhaling
\end{itemize}

Tenslotte is het belangrijk om te weten dat bij elderspeak de inhoud van de boodschap, die de zorgverlener wil overbrengen, niet wijzigt. Wel verandert de wijze waarop de boodschap wordt overgebracht door het gebruik van een infantiliserende communicatiestijl, aldus \autocite{TODO} TODO Campens

\subsection{Wat zijn de tips om \textit{elderspeak} te voorkomen?}

De onderstaande tips zijn enkele van de tips die Wick en Zanni (2007) meegeven ter “bestrijding” van \textit{elderspeak}:

\begin{itemize}
    \item Spreek personen aan met de naam waarmee ze willen aangesproken worden. Gebruik geen collectieve voornaamwoorden als die niet van toepassing zijn.
    \item Als een persoon een zorgverlener toelaat om hem of haar met zijn voornaam aan te spreken, ga er dan niet van uit dat deze “toestemming” voor alle zorgverleners geldt. De mate van intimiteit varieert, waardoor elke zorgverlener moet nagaan hoe hij of zij zijn gesprekspartner mag aanspreken.
    \item Vermijd het gebruik van troetelnamen en overmatige intieme liefkozingen, tenzij de gesprekspartner uitdrukkelijk aangeeft dat hij of zij zo wil aangesproken worden.
    \item Wees je bewust van je non-verbaal gedrag.
    \item Verhoog je stemvolume (in beperkte mate) enkel en alleen wanneer de gesprekspartner hardhorig is. Verhoging van je stemvolume betekent geen verhoging van je stemhoogte. Wees je ervan bewust dat niet elke oudere gesprekspartner hardhorig is.
    \item Herhaling en verminderde grammaticale complexiteit hebben een plaats als de gesprekspartner je niet begrepen heeft.
    \item Vermijd korte zinnen en langzaam en met overdreven intonatie uitgesproken zinnen.
    \item Vermijd het gebruik van verkleinwoorden, aangezien die het gevoel van afhankelijkheid kunnen versterken en denigrerend kunnen geïnterpreteerd worden.
    \item Vermijd overdreven directieve boodschappen en bied keuzevrijheid.
    \item Hanteer een beleefd taalgebruik en beleefde omgangsvormen (bv. op de kamerdeur kloppen alvorens de kamer binnen te gaan).
\end{itemize}
