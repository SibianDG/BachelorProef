%%=============================================================================
%% Voorwoord
%%=============================================================================

\chapter*{\IfLanguageName{dutch}{Woord vooraf}{Preface}}
\label{ch:voorwoord}

%% TODO:
%% Het voorwoord is het enige deel van de bachelorproef waar je vanuit je
%% eigen standpunt (``ik-vorm'') mag schrijven. Je kan hier bv. motiveren
%% waarom jij het onderwerp wil bespreken.
%% Vergeet ook niet te bedanken wie je geholpen/gesteund/... heeft

Ik wilde een onderwerp kiezen dat een impact heeft op de maatschappij en waarbij ik zowel artificiële intelligentie als websiteontwikkeling kon gebruiken. Enerzijds om te kunnen aantonen dat AI niet altijd een negatieve connotatie moet hebben en anderzijds omdat ik AI \& \textit{Data Engineering} had gekozen als afstudeerrichting. Op die manier kon ik ondervinden of dit iets voor mij is, of toch eerder het programmeren an sich$\ldots$ (\textit{spoiler}: het is programmeren geworden.)

Het was echt een interessant onderwerp! Aangezien er onze maatschappij aan het vergrijzen is, neemt de nood aan ouderenzorg toe. Er is daarbij een duidelijk verschil tussen zorg op papier en kwalitatieve zorg in het echte leven. Ik hoop dat studenten in de opleiding verpleegkunde dankzij mijn bachelorproef minder \textit{elderspeak} zullen gebruiken, zodat ouderen niet behandeld worden als kleine kinderen.

Als laatste puntje wil ik iedereen bedanken die me geholpen heeft met dit eindwerk. Zo heeft mijn zus Shauni de tekst meermaals gelezen en verbeterpuntjes aangehaald. Ook vrienden en familie die mijn scriptie hebben nagelezen wil ik bedanken!
Als laatste bedank ik mijn promotor, Geert van Boven, en mijn co-promotor, Jorrit Campens, om feedback, maar ook tips en uitleg te geven doorheen het hele proces!

Dankjewel aan iedereen, zonder jullie was dit niet zo goed gelukt!