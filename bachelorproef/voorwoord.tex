%%=============================================================================
%% Voorwoord
%%=============================================================================

\chapter*{\IfLanguageName{dutch}{Woord vooraf}{Preface}}
\label{ch:voorwoord}

%% TODO:
%% Het voorwoord is het enige deel van de bachelorproef waar je vanuit je
%% eigen standpunt (``ik-vorm'') mag schrijven. Je kan hier bv. motiveren
%% waarom jij het onderwerp wil bespreken.
%% Vergeet ook niet te bedanken wie je geholpen/gesteund/... heeft

De keuze aan bachelorproefonderwerpen was omvangrijk. Ik wilde een onderwerp kiezen die een impact heeft op de maatschappij én waarbij ik Artificiële Intelligentie kon gebruiken. Enerzijds om te kunnen aantonen dat AI niet altijd een negatieve connotatie moet hebben en anderzijds omdat mijn afstudeerrichting AI \& \textit{Data Engineering} was. Op die manier kon ik te weten komen of AI \& \textit{Data Engineering} eerder iets voor mij is, of toch eerder het programmeren zelf..

Het was best wel een interessant onderwerp! Aangezien er meer en meer vergrijzing komt, is er meer nood aan ouderenzorg. Er is daarbij een duidelijk verschil tussen zorg op papier en kwalitatieve zorg in het echte leven. Ik hoop dat studenten in de opleiding verpleegkunde hierdoor minder aan \textit{elderspeak} zullen doen zodat ouderen niet behandeld worden als kleine kinderen. Hopelijk zal dit niet het geval zijn wanneer ik oud zal zijn en zorg zal nodig hebben\ldots

%TODO: hoe is het verlopen?

Het ontvangen van audiosamples was wel heel moeilijk omdat het niet zomaar een vragenlijst invullen was door wat aan te klikken, maar er werd gevraagd om zinnen in te spreken. Dit zorgde dat mensen sneller afhaakten. Toch wil ik alle 25 %TODO
mensen bedanken die me geholpen hebben om mijn systeem te testen.

Eerst en vooral wil ik mijn hogeschool, HOGENT, bedanken om zo een interessant onderwerp aan te bieden voor mijn eindwerk. Ook het feit dat dit zicht uitstrekte over twee studierichtingen heen, namelijk Toegepaste Informatica en Verpleegkunde, was een bijzondere maar leuke opstelling.

Daarnaast wil ik mijn promotor Geert Van Boven bedanken om de inhoud verschillende keren na te lezen, om mij extra informatie te geven over alle IT-onderwerpen en mij te steunen in dit proces.
Daarbij wil ik ook mijn co-promotor bedanken voor de probleemstelling duidelijk uit te leggen, voor zijn enthousiasme en zijn feedback!

Als laatste wil ik nog mijn vrienden en familie bedanken om mijn inhoud na te lezen, te verbeteren en aan te duiden wat er niet zo duidelijk was.

Dankjewel aan iedereen, zonder jullie was dit niet zo goed gelukt!