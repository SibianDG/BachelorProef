%%=============================================================================
%% Voorwoord
%%=============================================================================

\chapter*{\IfLanguageName{dutch}{Woord vooraf}{Preface}}
\label{ch:voorwoord}

%% TODO:
%% Het voorwoord is het enige deel van de bachelorproef waar je vanuit je
%% eigen standpunt (``ik-vorm'') mag schrijven. Je kan hier bv. motiveren
%% waarom jij het onderwerp wil bespreken.
%% Vergeet ook niet te bedanken wie je geholpen/gesteund/... heeft



Ik wilde een onderwerp kiezen dat een impact heeft op de maatschappij en waarbij ik zowel artificiële intelligentie als websiteontwikkeling kon gebruiken. Enerzijds om te kunnen aantonen dat AI niet altijd een negatieve connotatie moet hebben en anderzijds omdat mijn afstudeerrichting AI \& \textit{Data Engineering} was. Op die manier kon ik te weten komen of AI \& \textit{Data Engineering} eerder iets voor mij is, of toch eerder het programmeren zelf$\ldots$ (\textit{spoiler}: het is het programmeren geworden.)

Het was best wel een interessant onderwerp! Aangezien er meer en meer vergrijzing komt, is er meer nood aan ouderenzorg. Er is daarbij een duidelijk verschil tussen zorg op papier en kwalitatieve zorg in het echte leven. Ik hoop dat studenten in de opleiding verpleegkunde hierdoor minder aan \textit{elderspeak} zullen doen zodat ouderen niet behandeld worden als kleine kinderen. Hopelijk zal dit niet het geval zijn wanneer ik oud zal zijn en zorg zal nodig hebben\ldots

Als laatste puntje wil ik iedereen bedanken die me geholpen heeft met dit eindwerk. Zo heeft mijn zus Shauni de tekst meermaals gelezen verbeterpuntjes aangehaald. Mijn vrienden en familie die mijn tekst hebben nagelezen wil ik ook bedanken!
Als laatste bedank ik mijn promotor, Geert van Boven, en mijn co-promotor, Jorrit Campens, voor feedback te geven doorheen het proces, maar ook om tips te geven en de uitleg!

Dankjewel aan iedereen, zonder jullie was dit niet zo goed gelukt!