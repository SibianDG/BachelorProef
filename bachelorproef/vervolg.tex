\chapter{Vervolg}
\label{ch:vervolg}

In dit hoofdstuk zal er beschreven worden wat het vervolg kan zijn van deze bachelorproef. Hoe kunnen anderen aan de programmeercode geraken, hoe kunnen anderen de website raadplegen of hoe kan de \textit{webserver} geïnstalleerd worden binnen het HoGent-netwerk.

\section{Applicatie-code delen}
Om er zeker van te zijn dat deze bachelorproef niet onder het stof verdwijnt en dat de applicatie niet meer gebruikt zal worden is het noodzakelijk dat de volledige programmacode gedeeld kan worden. Hiervoor zal er gebruik gemaakt worden van ``GitHub''.

Om GitHub te begrijpen is het eerst nodig wat Git zelf is. Git is een open source versiebeheersysteem. Wanneer ontwikkelaars iets maken of programmeren brengen ze wijzigingen aan in de code waarbij er nieuwe versies uitkomen. Versiebeheersystemen beheren deze revisies door alle wijzigingen op te slaan in een centrale opslagplaats. Hierdoor kunnen ontwikkelaars gemakkelijk samenwerken, omdat ze een nieuwe versie van de software kunnen downloaden, wijzigen en zelf nieuwe revisies kunnen uploaden. Elke programmeur van de project kan dan op zijn of haar beurt weer zien welke wijzigingen er doorgevoerd zijn en kan deze opnieuw downloaden.~\autocite{Brown2019}

GitHub is dus zo een voorbeeld van een centrale opslagplaats plaats waar alle wijzigingen opgeslagen staan in de \textit{cloud}.

Alle code, bestanden en revisies van deze volledige bachelorproef zijn dan ook te vinden op de volgende URL: ``\url{https://github.com/SibianDG/BachelorProef}'' met de nodige toestemmingen.

