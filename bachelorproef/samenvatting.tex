%%=============================================================================
%% Samenvatting
%%=============================================================================

% TODO: De "abstract" of samenvatting is een kernachtige (~ 1 blz. voor een
% thesis) synthese van het document.
%
% Deze aspecten moeten zeker aan bod komen:
% - Context: waarom is dit werk belangrijk?
% - Nood: waarom moest dit onderzocht worden?
% - Taak: wat heb je precies gedaan?
% - Object: wat staat in dit document geschreven?
% - Resultaat: wat was het resultaat?
% - Conclusie: wat is/zijn de belangrijkste conclusie(s)?
% - Perspectief: blijven er nog vragen open die in de toekomst nog kunnen
%    onderzocht worden? Wat is een mogelijk vervolg voor jouw onderzoek?
%
% LET OP! Een samenvatting is GEEN voorwoord!

%%---------- Nederlandse samenvatting -----------------------------------------
%
% TODO: Als je je bachelorproef in het Engels schrijft, moet je eerst een
% Nederlandse samenvatting invoegen. Haal daarvoor onderstaande code uit
% commentaar.
% Wie zijn bachelorproef in het Nederlands schrijft, kan dit negeren, de inhoud
% wordt niet in het document ingevoegd.

\IfLanguageName{english}{%
\selectlanguage{dutch}
\chapter*{Samenvatting}
\lipsum[1-4]
\selectlanguage{english}
}{}

%%---------- Samenvatting -----------------------------------------------------
% De samenvatting in de hoofdtaal van het document

\chapter*{\IfLanguageName{dutch}{Samenvatting}{Abstract}}

\color{blue}
TODO: De "abstract" of samenvatting is een kernachtige (~ 1 blz. voor een
thesis) synthese van het document.

Deze aspecten moeten zeker aan bod komen:
- Context: waarom is dit werk belangrijk?
- Nood: waarom moest dit onderzocht worden?
- Taak: wat heb je precies gedaan?
- Object: wat staat in dit document geschreven?
- Resultaat: wat was het resultaat?
- Conclusie: wat is/zijn de belangrijkste conclusie(s)?
- Perspectief: blijven er nog vragen open die in de toekomst nog kunnen
onderzocht worden? Wat is een mogelijk vervolg voor jouw onderzoek?

LET OP! Een samenvatting is GEEN voorwoord!

\color{black}
Deze bachelorproef gaat over het detecteren van \textit{elderspeak} a.d.h.v. een webapplicatie. \textit{Elderspeak} is het fenomeen waarbij jongere mensen tegen bejaarde mensen spreken op een betuttelde manier,  vandaar dat \textit{elderspeak} ook wel als \textit{secondary baby talk} wordt benoemd.

Nu dit onderzocht werd, kan deze applicatie gebruikt worden bij opleidingen in de zorgsector. De studenten kunnen dit leren in een practicum waarbij ze situaties moeten naspelen. De website zal dan aangeven of er \textit{elderspeak} aanwezig is.

Dit is belangrijk zodat senioren minder het gevoel hebben dat ze als kleine kinderen behandeld worden. Op die manier worden zij in hun waarde gelaten tijdens het communiceren met zorgpersoneel terwijl hun leven op z’n einde loopt.

Om delen van \textit{elderspeak} te herkennen in het spreken van een persoon, was de eerste stap het ijken van spraakherkenningssoftware met de kenmerken van \textit{elderspeak}. Spraakherkenning gebruikt achterliggend een geavanceerd artificieel intelligentie-model en daarboven op wordt \textit{natural language processing} of \textit{NLP} uitgevoerd om de precieze te verbeteren. Die spraakherkenning kan alleen werken als er zo weinig mogelijk achtergrondlawaai is. Deze drie technieken worden beschreven in het literatuuronderzoek.

Nadien beschrijft deze bachelorproef hoe de applicatie van nul af aan opgebouwd is, welke methoden en \textit{frameworks} gebruikt worden en hoe alles met elkaar verweven is. Zo is er gebruik gemaakt van een \textit{micro-framework} Flask in de programmeertaal Python om snel een webapplicatie op te zetten. Via de Google \textit{Speech Recognition}-API en andere Python-bibliotheken kan \textit{elderspeak} gedetecteerd worden.

In het hoofdstuk daarna worden de resultaten van het testen beschreven. Zo werd er data verzameld en werden deze gegeven aan een Python-script dat automatisch \textit{confusion matrixes} gebruikt om de accuraatheid van de applicatie weer te geven. Daarbij wordt er ook beschreven of alle \textit{requirements} voldaan zijn om de applicatie af te leven.

Als voorlaatste hoofdstuk wordt er beschreven hoe dit eindwerk kan verder leven en wat een mogelijk vervolg kan zijn in een eventuele interdisciplinaire omgeving. Via GitHub kan de code gedeeld worden en kan de applicatie gemakkelijk geïnstalleerd worden op een andere computer. Zo kan de applicatie draaien op een computer, maar via een parameter kunnen ook andere apparaten aan de website.

De conclusie van deze bachelorproef is dat het mogelijk is om \textit{elderspeak} te detecteren via AI, specifiek via spraakherkenning en Python-methoden. Dit is in een mooi jasje gegoten op een website zodat studenten in de zorg op dit fenomeen kunnen oefenen.

