%%=============================================================================
%% Samenvatting
%%=============================================================================

% TODO: De "abstract" of samenvatting is een kernachtige (~ 1 blz. voor een
% thesis) synthese van het document.
%
% Deze aspecten moeten zeker aan bod komen:
% - Context: waarom is dit werk belangrijk?
% - Nood: waarom moest dit onderzocht worden?
% - Taak: wat heb je precies gedaan?
% - Object: wat staat in dit document geschreven?
% - Resultaat: wat was het resultaat?
% - Conclusie: wat is/zijn de belangrijkste conclusie(s)?
% - Perspectief: blijven er nog vragen open die in de toekomst nog kunnen
%    onderzocht worden? Wat is een mogelijk vervolg voor jouw onderzoek?
%
% LET OP! Een samenvatting is GEEN voorwoord!

%%---------- Nederlandse samenvatting -----------------------------------------
%
% TODO: Als je je bachelorproef in het Engels schrijft, moet je eerst een
% Nederlandse samenvatting invoegen. Haal daarvoor onderstaande code uit
% commentaar.
% Wie zijn bachelorproef in het Nederlands schrijft, kan dit negeren, de inhoud
% wordt niet in het document ingevoegd.

\IfLanguageName{english}{%
\selectlanguage{dutch}
\chapter*{Samenvatting}
\lipsum[1-4]
\selectlanguage{english}
}{}

%%---------- Samenvatting -----------------------------------------------------
% De samenvatting in de hoofdtaal van het document

\chapter*{\IfLanguageName{dutch}{Samenvatting}{Abstract}}

\color{blue}
TODO: De "abstract" of samenvatting is een kernachtige (~ 1 blz. voor een
thesis) synthese van het document.

Deze aspecten moeten zeker aan bod komen:
- Context: waarom is dit werk belangrijk?
- Nood: waarom moest dit onderzocht worden?
- Taak: wat heb je precies gedaan?
- Object: wat staat in dit document geschreven?
- Resultaat: wat was het resultaat?
- Conclusie: wat is/zijn de belangrijkste conclusie(s)?
- Perspectief: blijven er nog vragen open die in de toekomst nog kunnen
onderzocht worden? Wat is een mogelijk vervolg voor jouw onderzoek?

LET OP! Een samenvatting is GEEN voorwoord!

\color{black}
Deze bachelorproef behandelt het detecteren van \textit{elderspeak} a.d.h.v. een webapplicatie. \textit{Elderspeak} is het fenomeen waarbij jongere mensen tegen ouderen spreken op een betuttelende manier,  vandaar dat \textit{elderspeak} ook wel als \textit{secondary baby talk} wordt benoemd.

Hopelijk kan deze applicatie gebruikt worden bij opleidingen in de zorgsector. De studenten kunnen dit leren in een practicum waarbij ze situaties moeten naspelen. De website zal dan aangeven of er \textit{elderspeak} aanwezig is.

Dit optimaliseert de communicatie tussen zorgpersoneel en ouderen, waardoor er kan voorzien worden in een kwaliteitsvolle zorg. Op die manier is er de mogelijkheid om een waardige levenseindefase te voorzien.

Om aspecten van \textit{elderspeak} te kunnen herkennen in een gesprek, was de eerste stap het ijken van spraakherkenningssoftware met de kenmerken van \textit{elderspeak}. Spraakherkenning gebruikt achterliggend een geavanceerd artificieel intelligentie-model en daarnaast wordt \textit{natural language processing} of \textit{NLP} toegepast om de precisie te verbeteren. De spraakherkenning kan enkel werken bij een minimum aan achtergrondlawaai. Deze drie technieken worden beschreven in het literatuuronderzoek.

Nadien beschrijft deze bachelorproef hoe de applicatie van scratch opgebouwd is, welke methoden en \textit{frameworks} gebruikt worden en hoe alles met elkaar verweven werd. Zo werd een \textit{micro-framework} Flask in de programmeertaal Python gebruikt om snel een webapplicatie op te zetten. Via de Google \textit{Speech Recognition}-API en andere Python-bibliotheken kan \textit{elderspeak} gedetecteerd worden.

In het volgende hoofdstuk worden de resultaten van het testen beschreven. Zo werd er data verzameld en werden geïmplementeerd in een Python-script dat automatisch \textit{confusion matrixes} gebruikt om de accuraatheid van de applicatie weer te geven. Daarbij wordt er ook beschreven of er aan alle \textit{requirements} voldaan is om de applicatie af te leveren.

Als voorlaatste hoofdstuk wordt er beschreven hoe dit eindwerk kan verder evolueren en wat een mogelijk vervolg kan zijn in een multidisciplinaire omgeving. Via GitHub kan de code gedeeld worden en kan de applicatie gemakkelijk geïnstalleerd worden op een andere computer. Zo kan de applicatie geraadpleegd op een computer, maar via een parameter kunnen ook andere apparaten gebruik maken van de website.

Deze bachelorproef besluit dat het mogelijk is om \textit{elderspeak} te detecteren via AI, specifiek via spraakherkenning en Python-methoden. Dit werd gevisualiseerd op een website zodat studenten in de zorgsector op beroep kunnen doen op deze tool om te oefenen.

