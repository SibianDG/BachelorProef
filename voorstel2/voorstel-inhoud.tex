%---------- Inleiding ---------------------------------------------------------

\section{Introductie} % The \section*{} command stops section numbering
\label{sec:introductie}

De veroudering van de bevolking in de Vlaamse steden en gemeenten zet zich in de komende 10 jaar verder \textcite{StatistiekVlaanderen2018}. Dit staat te lezen in een artikel van Statistiek Vlaanderen. Volgens hun voorspellingen zou tegen 2033 25\% van de bevolking een 65-plusser zijn~\autocite{StatistiekVlaanderen2018}.

Hoe meer ouderen er in de samenleving zijn, hoe meer zorg die natuurlijk gaan nodig hebben en hoe meer zorgverleners zich moeten bezig houden met die leeftijdscategorie. Die zorgverleners, maar evengoed familie of omstaanders weten niet altijd even goed hoe ze moeten omgaan met senioren.

Daarnaast is het woord 'waardigheid' actueler dan ooit. Na de schrijnende omstandigheden van de Tweede Wereldoorlog stond dat woord centraal bij het opstellen van het verdrag van de Verenigde Naties (1945), de Universele Verklaring van de Rechten van de mens (1948) en in de grondrechten van de Europese Unie (2000). Dat begrip 'waardigheid' wordt ook toegepast naar de zorgcontext. Dit wil zeggen dat ouderen niet gediscrimineerd mogen worden op vlak van leeftijd, maar ook dat ze niet op een kinderlijke, betuttelende of onvriendelijke wijze aangesproken worden en dat ze met respect bejegend moeten worden.~\autocite{Campens}

Wanneer een jonger persoon anders spreekt tegen een senior dan tegen een leeftijdsgenoot, spreken we over \textit{elderspeak}. \textcite{Williams2011} omschrijft \textit{elderspeak} als het volgende: `Elderspeak is a common intergenerational speech style used by younger persons in communication with older adults in a variety of community and health care settings. Based on negative stereotypes of older adults as less competent communicators, younger speakers (in this case nursing home staff) modify their communication with nursing home residents by simplifying the vocabulary and grammar and by adding clarifications such as repetitions and altered prosody.`. Om elderspeak te bestrijden gaven Wick en \textcite{Wick2007} een paar tips mee in hun onderzoek. Zo spreek je de persoon aan hoe ze aangesproken willen worden, vraag om ze aan te spreken met de voornaam, vermijd troetelnamen, wees bewust van non-verbaal gedrag, verhoog je stemvolume enkel en alleen wanneer de gesprekspartner hardhorig is, herhaal alleen je zin als de gesprekspartner het niet verstaan heeft, vermijd kort, langzame en makkelijke zinnen, vermijd verklein woorden en hanteer beleefd taalgebruik.

Dit onderwerp was vorig jaar al een onderzoeksonderwerp voor een eindwerk. Dit onderwerp werd gekozen door Glenn Beeckman~\autocite{Beeckman2021} en Victor Standaert~\autocite{Standaert2021}. Zij hebben al een basis gelegd in de goede richting om dit tot een goed einde te brengen. Ik kan dan ook sommige stukken programmeercode van hen gebruiken om zo een beter model op te stellen.

Met dit eindwerk of bachelorproef zal ik alle mogelijkheden en capaciteiten van mezelf inzetten om een applicatie én AI-model te maken zodat dit kan getest en gebruikt worden in de opleiding verpleegkunde.


Hier introduceer je werk. Je hoeft hier nog niet te technisch te gaan.

Je beschrijft zeker:

\begin{itemize}
  \item de probleemstelling en context
  \item de motivatie en relevantie voor het onderzoek
  \item de doelstelling en onderzoeksvraag/-vragen
\end{itemize}

%---------- Stand van zaken ---------------------------------------------------

\section{State-of-the-art}
\label{sec:state-of-the-art}

Momenteel is er al werk gemaakt van het speech-to-text, verkleinwoorden detecteren, een frequentiemeter, herhalende zinnen herkennen, emotie-herkenner en een basis applicatie in 'Tkinter', een standaard Graphical User Interface (GUI) in Python.

Het is de bedoeling dat de twee reeds vernoemde eindwerken samengevoegd en verbeterd worden. Flask, een micro webframework in Python, kan gebruikt worden om een webpagina te maken waar men kan oefenen op \textit{elderspeak}.

Ook het gebruiken van Machinaal leren of \textit{Deep Learning}, meer specifiek een \textit{Convolutional Neural Network} (CNN) kan een positief effect hebben op het detecteren van alle parameters rond \textit{elderspeak}.


Hier beschrijf je de \emph{state-of-the-art} rondom je gekozen onderzoeksdomein. Dit kan bijvoorbeeld een literatuurstudie zijn. Je mag de titel van deze sectie ook aanpassen (literatuurstudie, stand van zaken, enz.). Zijn er al gelijkaardige onderzoeken gevoerd? Wat concluderen ze? Wat is het verschil met jouw onderzoek? Wat is de relevantie met jouw onderzoek?

Verwijs bij elke introductie van een term of bewering over het domein naar de vakliteratuur, bijvoorbeeld~\autocite{Doll1954}! Denk zeker goed na welke werken je refereert en waarom.

% Voor literatuurverwijzingen zijn er twee belangrijke commando's:
% \autocite{KEY} => (Auteur, jaartal) Gebruik dit als de naam van de auteur
%   geen onderdeel is van de zin.
% \textcite{KEY} => Auteur (jaartal)  Gebruik dit als de auteursnaam wel een
%   functie heeft in de zin (bv. ``Uit onderzoek door Doll & Hill (1954) bleek
%   ...'')

Je mag gerust gebruik maken van subsecties in dit onderdeel.

%---------- Methodologie ------------------------------------------------------
\section{Methodologie}
\label{sec:methodologie}

Hier beschrijf je hoe je van plan bent het onderzoek te voeren. Welke onderzoekstechniek ga je toepassen om elk van je onderzoeksvragen te beantwoorden? Gebruik je hiervoor experimenten, vragenlijsten, simulaties? Je beschrijft ook al welke tools je denkt hiervoor te gebruiken of te ontwikkelen.

Voor dit onderzoek zal ik het kort hebben wat \textit{elderspeak} precies is. Dit zal een literatuuronderzoek zijn naar de reden waarom \textit{elderspeak} niet goed is, de kenmerken en hoe men dit kan helpen voorkomen.

Met deze kenmerken en samen met de twee eindwerken van vorig jaar kan ik methodes opstellen en verbeteren om die kenmerken te herkennen. Daarbij is het gebruik van Artificiële Intelligentie aan te raden om het verschil te weten tussen iemand die normaal praat en die aan \textit{elderspeak} doet.

Als laatste moet alles omgezet worden naar een duidelijk webapplicatie via \textit{Flask} zodat het in latere fases niet geïnstalleerd moet worden op de computer, maar wel zodat het door iedereen thuis of op verplaatsing, zoals een rusthuis, kan gebruikt worden.


%---------- Verwachte resultaten ----------------------------------------------
\section{Verwachte resultaten}
\label{sec:verwachte_resultaten}

Hier beschrijf je welke resultaten je verwacht. Als je metingen en simulaties uitvoert, kan je hier al mock-ups maken van de grafieken samen met de verwachte conclusies. Benoem zeker al je assen en de stukken van de grafiek die je gaat gebruiken. Dit zorgt ervoor dat je concreet weet hoe je je data gaat moeten structureren.

Het verwachte resultaat is een webapplicatie met \textit{Flask} als backend, waarbij men de optie heeft om het model te trainen en waarbij het model voorspelt wat of het \textit{elderspeak} is, met daarboven op dat het model aangeeft a.d.h.v. welke parameters het denkt dat het om \textit{elderspeak} gaat.


%---------- Verwachte conclusies ----------------------------------------------
\section{Verwachte conclusies}
\label{sec:verwachte_conclusies}

Hier beschrijf je wat je verwacht uit je onderzoek, met de motivatie waarom. Het is \textbf{niet} erg indien uit je onderzoek andere resultaten en conclusies vloeien dan dat je hier beschrijft: het is dan juist interessant om te onderzoeken waarom jouw hypothesen niet overeenkomen met de resultaten.

De conclusie bij de resultaten zou zijn dat er een meetbaar verschik is tussen personen die \textit{nursery tone} gebruiken t.o.v. mensen die normaal praten. Het model kan nooit 100\% werken. In de praktijk spreekt met sneller dialect tegen ouderen. De algoritmes zijn getraind op Algemeen Nederlands (AN) en niet op dialect. Die algoritmes zijn ook vaker getraind op Nederlanders i.p.v. op Belgen, waardoor sommige klanken minder goed herkend zullen worden. Ook het achtergrondlawaai zal een uitdaging worden om dat er uit te filteren.

