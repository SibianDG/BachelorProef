% Voor literatuurverwijzingen zijn er twee belangrijke commando's:
% \autocite{KEY} => (Auteur, jaartal) Gebruik dit als de naam van de auteur
%   geen onderdeel is van de zin.
% \textcite{KEY} => Auteur (jaartal)  Gebruik dit als de auteursnaam wel een
%   functie heeft in de zin (bv. ``Uit onderzoek door Doll & Hill (1954) bleek
%   ...'')

%---------- Inleiding ---------------------------------------------------------

\section{Introductie}\label{sec:introductie} % The \section*{} command stops section numbering

De veroudering van de bevolking in de Vlaamse steden en gemeenten zet zich in de komende 10 jaar verder.~\autocite{StatistiekVlaanderen2018} Dit staat te lezen in een artikel van Statistiek Vlaanderen uit 2018.
Volgens hun voorspellingen zou tegen 2033, 25\% van de bevolking een 65-plusser zijn.~\autocite{StatistiekVlaanderen2018}

Het woord 'waardigheid' actueler dan ooit.
Na de schrijnende omstandigheden van de Tweede Wereldoorlog stond dat woord centraal bij het opstellen van het verdrag van de Verenigde Naties (1945), de Universele Verklaring van de Rechten van de mens (1948) en in de grondrechten van de Europese Unie (2000).
Dat begrip 'waardigheid' wordt ook toegepast in de zorgcontext.
Dit wil zeggen dat ouderen niet gediscrimineerd mogen worden op vlak van leeftijd, maar ook dat ze niet op een kinderlijke, betuttelende of onvriendelijke wijze aangesproken worden en dat ze met respect bejegend moeten worden~\autocite{Campens}.

Hoe meer ouderen er in de samenleving zijn, hoe meer zorg zij nodig hebben en hoe meer zorgverleners zich moeten bezig houden met die leeftijdscategorie.
Die zorgverleners, maar evengoed familie of omstaanders weten niet altijd even goed hoe ze moeten omgaan met senioren.
Wanneer een jonger persoon anders spreekt tegen een senior dan tegen een leeftijdsgenoot, spreken we over \textit{elderspeak}. \textcite{Williams2011} omschrijft \textit{elderspeak} als het volgende: \enquote{Elderspeak is a common intergenerational speech style used by younger persons in communication with older adults in a variety of community and health care settings. Based on negative stereotypes of older adults as less competent communicators, younger speakers (in this case nursing home staff) modify their communication with nursing home residents by simplifying the vocabulary and grammar and by adding clarifications such as repetitions and altered prosody.} Om \textit{elderspeak} te bestrijden gaven Wick en \textcite{Wick2007} een paar tips mee in hun onderzoek.
Zo spreekt u de persoon aan hoe ze aangesproken willen worden, vraag om ze aan te spreken met de voornaam, vermijd troetelnamen, wees bewust van non-verbaal gedrag, verhoog uw stemvolume enkel en alleen wanneer de gesprekspartner hardhorig is, herhaal alleen uw zin als de gesprekspartner het niet verstaan heeft, vermijd kort, langzame en makkelijke zinnen, vermijd verklein woorden en hanteer beleefd taalgebruik.

Dit onderwerp was vorig jaar al een onderzoeksonderwerp voor een eindwerk.
Dit onderwerp werd gekozen door Glenn~\textcite{Beeckman2021} en Victor~\textcite{Standaert2021}.
Zij hebben al een basis gelegd in de goede richting om dit tot een goed einde te brengen.
Sommige stukken programmeercode van hen zullen gebruikt worden om zo een beter model op te stellen.
Zij haalden dan ook verbeterpunten aan en moeilijkheden die hopelijk op te lossen zijn.

De onderzoeksvraag die nog steeds relevant is voor dit eindwerk is: "Kan \textit{elderspeak} gedetecteerd worden door Artificiële Intelligentie en kan dit toegepast worden in de praktijk?"

Met dit eindwerk of bachelorproef zal ik alle mogelijkheden en capaciteiten van mezelf inzetten om een applicatie én AI-model te maken zodat dit kan getest en gebruikt worden in de opleiding verpleegkunde.
Ik hoop dat ouderen hiermee een betere communicatie zullen hebben met hun kinderen, familie en zorgverleners.
Dit was niet altijd het geval bij mijn grootouders.

\textcolor{blue}{Hier introduceer je werk. Je hoeft hier nog niet te technisch te gaan.
Je beschrijft zeker:
\begin{itemize}
  \item de probleemstelling en context
  \item de motivatie en relevantie voor het onderzoek
  \item de doelstelling en onderzoeksvraag/-vragen
\end{itemize}}


%---------- Stand van zaken ---------------------------------------------------

\section{State-of-the-art}
\label{sec:state-of-the-art}

\subsection{Literatuuronderzoek}\label{subsec:literatuuronderzoek}

Om écht te weten wat \textit{elderspeak} precies inhoudt, zal er eerst een kort literatuuronderzoek uitgevoerd moeten worden.
Dit zullen de belangrijkste vragen zijn om te beantwoorden:
\begin{itemize}
	\item Wat is \textit{elderspeak}?
	\item Waarom vinden senioren dat niet leuk?
	\item Wat zijn de kenmerken?
	\item Wat zijn tips om dat te verhelpen?
\end{itemize}

Ook zal er literatuuronderzoek gedaan moeten worden naar welke soort machineleren of \textit{deep learning} er het beste past bij al die verschillende parameters.
Of het mogelijk is om het model onmiddellijk te trainen en hoe dit moet worden toegepast in de \textit{back-end} van ``Flask''.

\subsection{Stand van zaken}\label{subsec:stand-van-zaken}

Zoals al vermeld in de inleiding hebben vorig jaar al twee studenten dit onderwerp gekregen voor hun bachelorproef.
Zij hebben werk gemaakt van het \textit{speech-to-text}, verkleinwoorden detecteren, een frequentiemeter, herhalende zinnen herkennen, emotie-herkenner en een basis applicatie in `Tkinter', een standaard \textit{Graphical User Interface} (GUI) in Python.

\textcite{Beeckman2021} vermeldde dat er nog werk was om herhaling en verkleinwoorden te detecteren. \textcite{Standaert2021} haalde aan dat er nog onderzoek nodig was voor de spraakherkenning en de frequentiemeter om de applicatie preciezer te maken.

\subsection{Wat is mijn aandeel TODO andere titel}\label{subsec:watismijnadeel}

Beide studenten hebben niet echt Kunstmatige Intelligentie gebruikt om het resultaat te bekomen. \textit{Standaert2021} heeft wel methoden geschreven om een paar kenmerken te herkennen, maar dat is op basis van vaste parameters.
Mocht AI gebruikt kunnen worden om de nauwkeurigheid te kunnen opschalen, dan zou dat alvast een winst zijn.
Het gebruik van Machinaal leren of \textit{Deep Learning}, meer specifiek een \textit{Convolutional Neural Network} (CNN) kan een positief effect hebben op het detecteren van alle parameters rond \textit{elderspeak}.

Het is de bedoeling dat de twee reeds vernoemde eindwerken samengevoegd en verbeterd worden. \textcite{Beeckman2021} gebruikte ``Tkinter'' voor de \textit{front-end} te maken, maar haalde een paar redenen aan waarom dat toch niet te verkiezen is waaronder het amateuristische uiterlijk en de beperkte mogelijkheden.
\textcolor{red}{Ik} zou dan verkiezen voor ``Flask'', een \textit{micro-webframework} in Python, dat kan gebruikt worden om een webpagina te maken en te linken naar de \textit{back-end}.
Het voordeel hiervan is dat men sneller én mooier een website kan ontwerpen.

\textcolor{blue}{
Hier beschrijf je de \emph{state-of-the-art} rondom je gekozen onderzoeksdomein. Dit kan bijvoorbeeld een literatuurstudie zijn. Je mag de titel van deze sectie ook aanpassen (literatuurstudie, stand van zaken, enz.). Zijn er al gelijkaardige onderzoeken gevoerd? Wat concluderen ze? Wat is het verschil met jouw onderzoek? Wat is de relevantie met jouw onderzoek?
Verwijs bij elke introductie van een term of bewering over het domein naar de vakliteratuur, bijvoorbeeld~\autocite{Doll1954}! Denk zeker goed na welke werken je refereert en waarom.
Je mag gerust gebruik maken van subsecties in dit onderdeel.}


%---------- Methodologie ------------------------------------------------------
\section{Methodologie}
\label{sec:methodologie}

\textcolor{blue}{Hier beschrijf je hoe je van plan bent het onderzoek te voeren. Welke onderzoekstechniek ga je toepassen om elk van je onderzoeksvragen te beantwoorden? Gebruik je hiervoor experimenten, vragenlijsten, simulaties? Je beschrijft ook al welke tools je denkt hiervoor te gebruiken of te ontwikkelen.}

Om er zeker van te zijn dat er genoeg data beschikbaar is, zal het beste zijn dat er audiosamples opgenomen worden voor de 1\textsuperscript{ste} examenperiode.
%TODO: 1e%

Met deze kenmerken die verzameld werden van het literatuuronderzoek en samen met de twee eindwerken van vorig jaar kunnen er methodes opgesteld en verbeterd worden die die kenmerken herkennen.
Daarbij is het gebruik van Artificiële Intelligentie aan te raden om het verschil te weten tussen iemand die normaal praat en die aan \textit{elderspeak} doet.

Als laatste moet alles omgezet worden naar een duidelijke webapplicatie via ``Flask'' zodat het in latere fases niet geïnstalleerd moet worden op de computer, maar wel zodat het door iedereen thuis of op verplaatsing, zoals een rusthuis, kan gebruikt worden.


%---------- Verwachte resultaten ----------------------------------------------
\section{Verwachte resultaten}
\label{sec:verwachte_resultaten}

\textcolor{blue}{Hier beschrijf je welke resultaten je verwacht. Als je metingen en simulaties uitvoert, kan je hier al mock-ups maken van de grafieken samen met de verwachte conclusies. Benoem zeker al je assen en de stukken van de grafiek die je gaat gebruiken. Dit zorgt ervoor dat je concreet weet hoe je je data gaat moeten structureren.}

%TODO: herschrijven
Het verwachte resultaat is een webapplicatie met ``Flask'' als \textit{back-end}, waarbij men de optie heeft om het model te trainen en waarbij het model voorspelt of er \textit{elderspeak} aanwezig is, met daarboven op dat het aangeeft a.d.h.v.\ welke parameters het `denkt' dat het om \textit{elderspeak} gaat.


%---------- Verwachte conclusies ----------------------------------------------
\section{Verwachte conclusies}
\label{sec:verwachte_conclusies}

\textcolor{blue}{Hier beschrijf je wat je verwacht uit je onderzoek, met de motivatie waarom. Het is \textbf{niet} erg indien uit je onderzoek andere resultaten en conclusies vloeien dan dat je hier beschrijft: het is dan juist interessant om te onderzoeken waarom jouw hypothesen niet overeenkomen met de resultaten.}

De conclusie bij de resultaten kan zijn dat er een meetbaar verschil is tussen personen die \textit{nursery tone} gebruiken t.o.v.\ mensen die normaal praten.
Het model kan nooit 100\% werken.
In de praktijk spreekt met sneller dialect tegen ouderen.
De algoritmes zijn trouwens getraind op Algemeen Nederlands (AN) en niet op dialect.
Die modellen zijn ook vaker getraind op Nederlanders i.p.v.\ op Vlamingen, waardoor sommige klanken minder goed herkend zullen worden.
Ook het achtergrondlawaai zal een uitdaging worden om dat er uit te filteren.
