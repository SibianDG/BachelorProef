%==============================================================================
% Sjabloon onderzoeksvoorstel bachelorproef
%==============================================================================
% Gebaseerd op LaTeX-sjabloon ‘Stylish Article’ (zie voorstel.cls)
% Auteur: Jens Buysse, Bert Van Vreckem
%
% Compileren in TeXstudio:
%
% - Zorg dat Biber de bibliografie compileert (en niet Biblatex)
%   Options > Configure > Build > Default Bibliography Tool: "txs:///biber"
% - F5 om te compileren en het resultaat te bekijken.
% - Als de bibliografie niet zichtbaar is, probeer dan F5 - F8 - F5
%   Met F8 compileer je de bibliografie apart.
%
% Als je JabRef gebruikt voor het bijhouden van de bibliografie, zorg dan
% dat je in ``biblatex''-modus opslaat: File > Switch to BibLaTeX mode.

\documentclass{voorstel}
\usepackage[dutch]{babel}
\usepackage{lipsum}
\usepackage{xcolor}
\usepackage[T1]{fontenc}
\usepackage{csquotes}

%------------------------------------------------------------------------------
% Metadata over het voorstel
%------------------------------------------------------------------------------

%---------- Titel & auteur ----------------------------------------------------

\PaperTitle{Nursery Tone Monitor: detecteren van elderspeak via AI}
\PaperType{Onderzoeksvoorstel Bachelorproef 2021--2022} % Type document

\Authors{Sibian {De Gussem}\textsuperscript{1}} % Authors
\CoPromotor{Jorrit {Campens}\textsuperscript{2} (Hogeschool Gent)}
\affiliation{\textbf{Contact:}
  \textsuperscript{1} \href{mailto:sibian.degussem@student.hogent.be}{sibian.degussem@student.hogent.be};
  \textsuperscript{2} \href{mailto:jorrit.campens@hogent.be}{jorrit.campens@hogent.be};
}

%---------- Abstract ----------------------------------------------------------

\Abstract{Deze bachelorproef is het vervolg op twee eerder gepubliceerde bachelorproef-onderzoeken van \textcite{Beeckman2021} en \textcite{Standaert2021} en zal op een softwarematige manier nagaan of er bij een stukje audio \textit{elderspeak} wordt toegepast. Een webapplicatie zal aanduiden of er verkleinwoorden of troetelnamen aanwezig waren, of er veel herhaald werd en of de toonhoogte hoger is dan anders. Ook zal er aangeduid worden of er verhoogd stemvolume aanwezig was. Dit zal gerealiseerd worden met Python, spraakherkenning, herkenningsmethoden, \textit{Machine Learning} of \textit{Deep Learning} en gepresenteerd worden via een webapplicatie met ``Flask'' als \textit{back-end}.
}

%---------- Onderzoeksdomein en sleutelwoorden --------------------------------
%
% Het eerste sleutelwoord beschrijft het onderzoeksdomein. Je kan kiezen uit
% deze lijst:
%
% - Mobiele applicatieontwikkeling
% - Webapplicatieontwikkeling
% - Applicatieontwikkeling (andere)
% - Systeembeheer
% - Netwerkbeheer
% - Mainframe
% - E-business
% - Databanken en big data
% - Machineleertechnieken en kunstmatige intelligentie
% - Andere (specifieer)
%
% De andere sleutelwoorden zijn vrij te kiezen

\Keywords{Machineleertechnieken en kunstmatige intelligentie, Artificiële Intelligentie, Elderspeak, spraakherkenning, verpleegkunde, Python, applicatieontwikkeling, zorg } % Keywords
\newcommand{\keywordname}{Sleutelwoorden} % Defines the keywords heading name

%---------- Titel, inhoud -----------------------------------------------------

\begin{document}

\flushbottom % Makes all text pages the same height
\maketitle % Print the title and abstract box
\tableofcontents % Print the contents section
\thispagestyle{empty} % Removes page numbering from the first page

%------------------------------------------------------------------------------
% Hoofdtekst
%------------------------------------------------------------------------------

% De hoofdtekst van het voorstel zit in een apart bestand, zodat het makkelijk
% kan opgenomen worden in de bijlagen van de bachelorproef zelf.
% Voor literatuurverwijzingen zijn er twee belangrijke commando's:
% \autocite{KEY} => (Auteur, jaartal) Gebruik dit als de naam van de auteur
%   geen onderdeel is van de zin.
% \textcite{KEY} => Auteur (jaartal)  Gebruik dit als de auteursnaam wel een
%   functie heeft in de zin (bv. ``Uit onderzoek door Doll & Hill (1954) bleek
%   ...'')

%---------- Inleiding ---------------------------------------------------------

\section{Introductie}\label{sec:introductie} % The \section*{} command stops section numbering

De veroudering van de bevolking in de Vlaamse steden en gemeenten zet zich in de komende  decennia verder.~\autocite{StatistiekVlaanderen2018}
Volgens hun voorspellingen zou tegen 2033, 25\% van de bevolking een 65-plusser zijn.

Het woord 'waardigheid' is actueler dan ooit.
Na de schrijnende omstandigheden van de Tweede Wereldoorlog stond dat woord centraal bij het opstellen van het verdrag van de Verenigde Naties (1945), de Universele Verklaring van de Rechten van de mens (1948) en in de grondrechten van de Europese Unie (2000).
Die basiswaarde vinden we ook terug bij ons Europese en nationale zorgbeleid. Ouderen mogen niet gediscrimineerd worden op vlak van leeftijd, maar ook dat ze niet op een kinderlijke, betuttelende of onvriendelijke wijze aangesproken worden en dat ze met respect bejegend moeten worden~\autocite{Campens}.

Hoe meer ouderen er in de samenleving zijn, hoe meer zorg zij nodig hebben en hoe meer zorgverleners instaan voor die leeftijdscategorie.
Die zorgverleners, maar evengoed familie of omstaanders weten niet altijd even goed hoe ze moeten omgaan met senioren.
Wanneer een jonger persoon op een andere manier spreekt tegen een senior dan tegen een leeftijdsgenoot, spreken we over \textit{elderspeak}. \textcite{Williams2011} omschrijft \textit{elderspeak} als het volgende: \enquote{Elderspeak is a common intergenerational speech style used by younger persons in communication with older adults in a variety of community and health care settings. Based on negative stereotypes of older adults as less competent communicators, younger speakers (in this case nursing home staff) modify their communication with nursing home residents by simplifying the vocabulary and grammar and by adding clarifications such as repetitions and altered prosody.} Om \textit{elderspeak} te bestrijden gaven Wick en \textcite{Wick2007} een paar tips mee in hun onderzoek.
Zo spreekt u de persoon aan hoe ze aangesproken willen worden, vraag om ze aan te spreken met de voornaam, vermijd troetelnamen, wees bewust van non-verbaal gedrag, verhoog uw stemvolume enkel en alleen wanneer de gesprekspartner hardhorig is, herhaal alleen uw zin als de gesprekspartner het niet verstaan heeft, vermijd korte, langzame en makkelijke zinnen, vermijd verkleinwoorden en hanteer beleefd taalgebruik.

Naast \textit{elderspeak} heb je ook nog \textit{nersery tone}. Dit verwijst naast de situatie waarbij iemand de toonhoogte aan het einde van de zin standaard verhoogt. Bijv.: de communicatie met peuters en kinderen met volwassenen.

Dit onderwerp was vorig jaar al een onderzoeksonderwerp voor een eindwerk.
Dit onderwerp werd gekozen door Glenn~\textcite{Beeckman2021} en Victor~\textcite{Standaert2021}.
Zij hebben al een basis gelegd in de goede richting om dit tot een goed einde te brengen.
Sommige stukken programmacode van hen zullen gebruikt worden om zo een beter model op te stellen.
Zij haalden dan ook verbeterpunten aan en moeilijkheden die, hopelijk, op te lossen zijn.

De onderzoeksvraag die nog steeds relevant is voor dit eindwerk is: ``Kan \textit{elderspeak} gedetecteerd worden door Artificiële Intelligentie en kan dit toegepast worden in de praktijk?''. Een bijkomende onderzoeksvraag is: ``Kan \textit{nersery tone} gedetecteerd worden door Artificiële Intelligentie?''.

Met dit eindwerk of deze bachelorproef zal ik alle mogelijkheden en capaciteiten van mezelf inzetten om een applicatie én AI-model te maken zodat dit kan getest en gebruikt worden in de opleiding verpleegkunde.
Ik hoop dat ouderen hiermee een betere communicatie zullen hebben met zorgverleners, en misschien zelfs hun kinderen en familie.


%---------- Stand van zaken ---------------------------------------------------

\section{State-of-the-art}
\label{sec:state-of-the-art}

\subsection{Literatuuronderzoek}\label{subsec:literatuuronderzoek}

Om écht te weten wat \textit{elderspeak} precies inhoudt, zal er eerst een kort literatuuronderzoek uitgevoerd moeten worden.
Dit zullen de belangrijkste vragen zijn om te beantwoorden:
\begin{itemize}
	\item Wat is \textit{elderspeak}?
	\item Waarom storen senioren zich aan \textit{elderspeak}?
	\item In welke context komt dit het meest naar boven?
	\item Wat zijn tips om dat te verhelpen?
\end{itemize}

Ook zal er literatuuronderzoek moeten worden gedaan naar welke soort machineleren of \textit{deep learning} er het beste past bij al die verschillende parameters.
Of het mogelijk is om het model onmiddellijk te trainen en hoe dit moet worden toegepast in de \textit{back-end} van ``Flask''.

\subsection{Stand van zaken}\label{subsec:stand-van-zaken}

Zoals al vermeld in de inleiding hebben vorig jaar al twee studenten dit onderwerp gekozen voor hun bachelorproef.
Zij hebben werk gemaakt van het \textit{speech-to-text}, verkleinwoorden detecteren, een frequentiemeter, herhalende zinnen herkennen, emotie-herkenner en een basis applicatie in `Tkinter', een standaard \textit{Graphical User Interface} (GUI) in Python.

\textcite{Beeckman2021} vermeldde dat er nog werk was om herhaling en verkleinwoorden te detecteren. \textcite{Standaert2021} haalde aan dat er nog onderzoek nodig was voor de spraakherkenning en de frequentiemeter om de applicatie preciezer te maken.

\subsection{Wat is mijn aandeel?}\label{subsec:watismijnadeel}

Beide studenten hebben niet echt Kunstmatige Intelligentie gebruikt om het resultaat te bekomen. \textit{Standaert2021} heeft wel methoden beschreven om een paar kenmerken te herkennen, maar dit gebeurt op basis van vaste parameters.
Mocht AI gebruikt kunnen worden om de nauwkeurigheid te kunnen opschalen, dan zou dat alvast een winst zijn.
Het gebruik van Machinaal leren of \textit{Deep Learning}, meer specifiek een \textit{Convolutional Neural Network} (CNN) kan een positief effect hebben op het detecteren van alle parameters rond \textit{elderspeak}.

Het doe is dat de twee reeds vernoemde eindwerken samengevoegd en verbeterd worden. \textcite{Beeckman2021} gebruikte ``Tkinter'' om de \textit{front-end} te maken, maar haalde een paar redenen aan waarom dat toch niet te verkiezen is, zoals bijvoorbeeld het amateuristische uiterlijk en de beperkte mogelijkheden.
Mijn voorkeur gaat eerder uit naar het gebruik van ``Flask'', een \textit{micro-webframework} in Python, dat kan gebruikt worden om een webpagina te maken en te linken naar de \textit{back-end}.
Het voordeel hiervan is dat men sneller én mooier een website kan ontwerpen.


%---------- Methodologie ------------------------------------------------------
\section{Methodologie}
\label{sec:methodologie}

Om er zeker van te zijn dat er genoeg data beschikbaar is, is aan te raden dat er audiosamples opgenomen worden voor de 1\textsuperscript{ste} examenperiode.

Met deze kenmerken die verzameld werden uit het literatuuronderzoek en samen met de twee eindwerken van vorig jaar kunnen er methodes opgesteld en verbeterd worden die die kenmerken herkennen.
Daarbij is het gebruik van Artificiële Intelligentie aan te raden om het verschil te kennen tussen iemand die aan \textit{elderspeak} doet en iemand die dat niet doet.

Als laatste moet alles omgezet worden naar een duidelijke webapplicatie via ``Flask'' zodat het in latere fases niet geïnstalleerd moet worden op de computer, maar wel zodat het door iedereen thuis of op verplaatsing, zoals een rusthuis, kan gebruikt worden.


%---------- Verwachte resultaten ----------------------------------------------
\section{Verwachte resultaten}
\label{sec:verwachte_resultaten}

Het verwachte resultaat is een webapplicatie met ``Flask'' als \textit{back-end}, waarbij men de optie heeft om het model te trainen, en waarbij het model aan geeft of er \textit{elderspeak} of \textit{nursery tone} aanwezig is. Bovendien geeft het aan op basis van welke parameters het model `denkt' dat het om die twee taalregisters gaat.


%---------- Verwachte conclusies ----------------------------------------------
\section{Verwachte conclusies}
\label{sec:verwachte_conclusies}

De conclusie bij de resultaten kan zijn dat er een meetbaar verschil is tussen personen die \textit{nursery tone} gebruiken t.o.v.\ mensen die normaal praten.
Het model kan nooit 100\% werken: zo spreekt met in de praktijk sneller dialect tegen ouderen terwijl de algoritmes getraind zijn op Algemeen Nederlands (AN), en ook het verschil tussen de Nederlandse uitspraak en de Vlaamse uitspraak kunnen een obstakel vormen
Ook het filteren van achtergrondlawaai wordt een bijkomende uitdaging.



%------------------------------------------------------------------------------
% Referentielijst
%------------------------------------------------------------------------------
% de gerefereerde werken moeten in BibTeX-bestand ``voorstel.bib''
% voorkomen. Gebruik JabRef om je bibliografie bij te houden en vergeet niet
% om compatibiliteit met Biber/BibLaTeX aan te zetten (File > Switch to
% BibLaTeX mode)

\phantomsection
\printbibliography[heading=bibintoc]

\end{document}
