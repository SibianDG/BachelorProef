% Voor literatuurverwijzingen zijn er twee belangrijke commando's:
% \autocite{KEY} => (Auteur, jaartal) Gebruik dit als de naam van de auteur
%   geen onderdeel is van de zin.
% \textcite{KEY} => Auteur (jaartal)  Gebruik dit als de auteursnaam wel een
%   functie heeft in de zin (bv. ``Uit onderzoek door Doll & Hill (1954) bleek
%   ...'')

%---------- Inleiding ---------------------------------------------------------

\section{Introductie}\label{sec:introductie} % The \section*{} command stops section numbering

De veroudering van de bevolking in de Vlaamse steden en gemeenten zet zich in de komende  decennia verder.~\autocite{StatistiekVlaanderen2018}
Volgens hun voorspellingen zou tegen 2033, 25\% van de bevolking een 65-plusser zijn.

Het woord 'waardigheid' is actueler dan ooit.
Na de schrijnende omstandigheden van de Tweede Wereldoorlog stond dat woord centraal bij het opstellen van het verdrag van de Verenigde Naties (1945), de Universele Verklaring van de Rechten van de mens (1948) en de grondrechten van de Europese Unie (2000).
Die basiswaarde vinden we ook terug bij het Europese en Belgische zorgbeleid. Ouderen mogen niet gediscrimineerd worden op vlak van leeftijd. Tevens mogen ze ook niet op een kinderlijke, betuttelende of onvriendelijke wijze aangesproken worden en moeten ze met respect bejegend worden~\autocite{Campens}.

Hoe meer ouderen er in de samenleving zijn, hoe meer zorg zij nodig hebben en hoe meer zorgverleners instaan voor deze leeftijdscategorie.
Die zorgverleners, maar evengoed familie, weten niet altijd even goed hoe ze moeten omgaan met senioren.
Wanneer een jonger persoon op een andere manier spreekt tegen een senior dan tegen een leeftijdsgenoot, spreken we over \textit{elderspeak}. \textcite{Williams2011} omschrijft \textit{elderspeak} als volgt: ``Elderspeak is a common intergenerational speech style used by younger persons in communication with older adults in a variety of community and health care settings. Based on negative stereotypes of older adults as less competent communicators, younger speakers (in this case nursing home staff) modify their communication with nursing home residents by simplifying the vocabulary and grammar and by adding clarifications such as repetitions and altered prosody.'' Om \textit{elderspeak} te bestrijden, gaven \textcite{Wick2007} een paar tips mee in hun onderzoek.
Spreek mensen aan zoals ze wensen aangesproken te worden, vraag om ze aan te spreken met de voornaam, vermijd troetelnamen, wees bewust van non-verbaal gedrag, verhoog uw stemvolume enkel wanneer uw gesprekspartner hardhorig is, herhaal alleen uw zin als uw gesprekspartner het niet begrepen heeft, vermijd korte, langzame en makkelijke zinnen, vermijd verkleinwoorden en hanteer beleefd taalgebruik.

Naast \textit{elderspeak} heb je ook nog \textit{nursery tone}. Dit verwijst naar de situatie waarbij iemand de toonhoogte aan het einde van de zin standaard verhoogt zoals bij communicatie met jonge kinderen.

Dit onderwerp was vorig jaar al een onderzoeksonderwerp voor een eindwerk.
Dit onderwerp werd gekozen door Glenn~\textcite{Beeckman2021} en Victor~\textcite{Standaert2021}.
Zij hebben al een basis gelegd in de goede richting om dit project tot een goed einde te brengen.
Sommige stukken programmacode van hen zullen gebruikt worden om zo een beter model op te stellen.
Zij haalden zelf ook verbeterpunten aan en moeilijkheden die, hopelijk, op te lossen zijn.

De onderzoeksvraag die nog steeds relevant is voor dit eindwerk is: ``Kan \textit{elderspeak} gedetecteerd worden door Artificiële Intelligentie en kan dit toegepast worden in de praktijk?''. Een bijkomende onderzoeksvraag is: ``Kan \textit{nursery tone} gedetecteerd worden door Artificiële Intelligentie?''.

Met dit eindwerk zal ik alle mogelijkheden en capaciteiten van mezelf inzetten om een applicatie én AI-model te maken zodat dit kan getest en gebruikt worden in de opleiding verpleegkunde.
Ik hoop ook dat ik ouderen op deze manier een betere levenskwaliteit kan bieden door de communicatie met zorgverleners, en misschien zelfs hun familie, te optimaliseren.


%---------- Stand van zaken ---------------------------------------------------

\section{State-of-the-art}
\label{sec:state-of-the-art}

\subsection{Literatuuronderzoek}\label{subsec:literatuuronderzoek}

Omdat voorgaande studenten al uitgezocht hebben wat \textit{elderspeak} precies is, zal dit niet herhaald worden in dit onderzoek.
Wel zal er op basis van de beschikbare literatuur onderzocht worden welk soort machinaal leren of \textit{deep learning} het meest geschikt is. Zowel \textit{machine learning} als \textit{deep learning} hebben elk verschillende onderlinge modellen. Er moet dan bekeken worden welke hypothese het beste past om deze parameters te integreren in het model of verschillende modellen.

Een extra obstakel kan verschijnen wanneer er te veel achtergrond lawaai aanwezig is. Mogelijks moet er dan eerst een filter worden uitgevoerd op de audiobestanden om dit weg te filteren zodat deze wel gebruikt kunnen worden voor het herkennen van eigenschappen op \textit{elderspeak}.

\subsection{Stand van zaken}\label{subsec:stand-van-zaken}

Zoals reeds vermeld in de inleiding werd dit bachelorproef-onderwerk vorig jaar al gekozen door twee studenten.
Zij hebben zich gefocust op de \textit{speech-to-text}, verkleinwoorden detecteren, een frequentiemeter, herhalende zinnen herkennen, emotie-herkenner en een basisapplicatie in `Tkinter', een standaard \textit{Graphical User Interface} (GUI) in Python.

\textcite{Beeckman2021} vermeldde dat er nog nood was aan een methode om herhaling en verkleinwoorden te detecteren. ~\textcite{Standaert2021} haalde aan dat er nog onderzoek nodig was voor de spraakherkenning en de frequentiemeter om de applicatie preciezer te maken.

\subsection{Wat is mijn aandeel?}\label{subsec:watismijnadeel}

Beide studenten hebben niet echt Kunstmatige Intelligentie gebruikt om het resultaat te bekomen. \textcite{Standaert2021} heeft wel methoden beschreven om een paar kenmerken te herkennen, maar dit gebeurt op basis van vaste parameters.
Mocht AI gebruikt kunnen worden om de nauwkeurigheid op te schalen, dan zou dat alvast een winst zijn.
Het gebruik van Machinaal leren of \textit{Deep Learning}, meer specifiek een \textit{Convolutional Neural Network} (CNN) kan een positief effect hebben op het detecteren van alle parameters rond \textit{elderspeak}. Mijn aandeel zal dus zijn om te onderzoeken welke modellen het beste gebruikt worden om die parameters te detecteren.

Het doel is om de twee eerder vernoemde eindwerken samen te voegen en te verbeteren. \textcite{Beeckman2021} gebruikte ``Tkinter'' om de \textit{front-end} te maken, maar haalde een paar redenen aan waarom dat toch niet te verkiezen is, zoals bijvoorbeeld het amateuristische uiterlijk en de beperkte mogelijkheden.
Mijn voorkeur gaat eerder uit naar het gebruik van ``Flask'', een \textit{micro-webframework} in Python, dat kan gebruikt worden om een webpagina te maken en te linken naar de \textit{back-end}.
Het voordeel hiervan is dat men sneller én mooier een website kan ontwerpen.


%---------- Methodologie ------------------------------------------------------
\section{Methodologie}
\label{sec:methodologie}

Om te verzekeren dat er genoeg data beschikbaar is, is het aan te raden dat er audiosamples verzameld worden voor het 2\textsuperscript{e} semester.

Op basis van de resultaten van het literatuuronderzoek en beide eindwerken van vorig jaar, kunnen er methodes opgesteld worden die de belangrijkste kenmerken van \textit{nursery tone} en \textit{eldery speak} herkennen.
Daarbij is het gebruik van Artificiële Intelligentie een handige manier om het verschil te kennen tussen iemand die \textit{elderspeak} gebruikt en iemand die dat niet doet. Met welke model en op welke wijze dit het beste gerealiseerd wordt, zal onderzocht worden in dit eindwerk.

Daarnaast moet alles omgezet worden naar een duidelijke webapplicatie via ``Flask'' zodat het in latere fases niet geïnstalleerd moet worden op een computer. Zo kan iedereen de applicatie gebruiken zonder vooraf iets te downloaden, wat het gebruiksgemak thuis en op verplaatsing, bijvoorbeeld in een rusthuis, optimaliseert.

Tot slotte zullen het beantwoorden van de volgende deelvragen hierbij moeten helpen:
\begin{itemize}
	\item Welk type Artificiële Intelligentie past het beste bij deze opstelling?
	\item Welk type model van \textit{machine learning} of \textit{deep learning} werkt het beste per eigenschap?
	\item Kan je achtergrond lawaai wegfilteren en hoe precies?
	\item Zal spraakherkenning lukken met de gratis beschikbare softwarebibliotheken?
	\item Hoe zet je een \{Flask} server op waar je {webrequests} naar stuurt? En hoe verbind je daar een model mee?
\end{itemize}


%---------- Verwachte resultaten ----------------------------------------------
\section{Verwachte resultaten}
\label{sec:verwachte_resultaten}

Het verwachte resultaat is een webapplicatie met ``Flask'' als \textit{back-end}, waarbij men de optie heeft om het model te trainen, en waarbij het model aangeeft of er \textit{elderspeak} of \textit{nursery tone} aanwezig is. Bovendien geeft de applicatie weer op basis van welke parameters het model `denkt' dat het om die twee taalregisters gaat.

%---------- Verwachte conclusies ----------------------------------------------
\section{Verwachte conclusies}
\label{sec:verwachte_conclusies}

De gehoopte resultaten houden in dat er een meetbaar verschil is tussen personen die \textit{nursery tone} gebruiken t.o.v.\ mensen die normaal praten.
Het model zal nooit 100\% accuraat zijn: zo spreekt men in de praktijk vaker dialect tegen ouderen terwijl de algoritmes getraind zijn op Algemeen Nederlands (AN), en ook het verschil tussen de Nederlandse uitspraak en de Vlaamse uitspraak kunnen een obstakel vormen.
Ook het filteren van achtergrondlawaai wordt een bijkomende uitdaging.

